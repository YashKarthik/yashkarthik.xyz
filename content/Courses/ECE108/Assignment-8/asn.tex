%%%%%%%%%%%%%%%%%%%%%%%%%%%%%%%%%%%%%%%%%% Preamble %%%%%%%%%%%%%%%%%%%%%%%%%%%%%%%%%%%%%%%%%%

\documentclass{assignment}
\usepackage[pdftex]{graphicx} % FIGURAS
\usepackage{xcolor}
\definecolor{LightGray}{gray}{0.95}
\usepackage[letterpaper, margin = 2.5cm]{geometry} % TAMAÑO DE PÁGINA Y MÁRGENES
\usepackage[T1]{fontenc} % Importante para acentos automáticos y símbolos de escritura
\usepackage{amsmath, amsfonts, amssymb} % Ecuaciones, caracteres y símbolos especiales
\usepackage{hyperref, url}
\usepackage{fancyhdr}

\def\multichoose#1#2{\ensuremath{\left(\kern-.3em\left(\genfrac{}{}{0pt}{}{#1}{#2}\right)\kern-.3em\right)}}

%-----------------------------------------ETIQUETAS--------------------------------------------

\student{Yashashwin Karthikeyan, Roozbeh Ali}
\semester{Winter 2024}
\date{\today}

\courselabel{ECE108}
\exercisesheet{Assignment 8}{Combinatorics}

\school{Dept. of ECE}
\university{University of Waterloo}

  %%%%%%%%%%%%%%%%%%%%%%%%%%%%%%%%%%%%%%%%%%-DOCUMENTO-%%%%%%%%%%%%%%%%%%%%%%%%%%%%%%%%%%%%%%%%%%%%

\begin{document}

  % Question 1
  \begin{problem}
    \section{Password Generator Configuration}
      The OS constraint for having at least two characters of each set (upper case, lower case
      numbers, special characters, numbers) makes sense since humans are susceptible to creating
      passwords that are easy to remember. We define \textit{easy to remember} passwords as those
      comprised of only english words (upper and lower case characters in orders that make up real
      words and not just arbitrary arrangements). These passwords are easy to brute force since the
      permutations that make up real english words is a small number.

      Having a similar constraint for password generators makes the generators less secure as the
      constraint reduces the number of possible passwords. The constraint does not help the
      generators in any meaningful way since the permutation of alphabets resulting in real words is
      very small, hence unlikely to be generated by the password generators.
  \end{problem}

  \begin{problem}
    \section{Co-op Team}
      Since the order of picking students for the team does not matter and we cannot pick the same
      student twice, we shall use combinations.

      \begin{flalign*}
        & = \binom{7}{6} \cdot \binom{3}{1}^6 &\\
        & = 7\cdot 3^6 &\\
        & = 5103 \text{ways of forming the coop team.} &
      \end{flalign*}

      Since we can only pick one student from each school, we use $\binom 3 1$ six times to pick six
      students. And we also have to pick the 6 school that will supply the students, $\binom 7 6$.
  \end{problem}

  \begin{problem}
    \section{Routes to Starbucks}
      Since I can only move $\big\uparrow$ 6 times and $\longrightarrow$ 10 times to get to the Starbucks store, we
      want to know the number of arrangements of $\big\uparrow$ and $\longrightarrow$ that get me to
      the store.

      \begin{flalign*}
        & = \binom{16}{10, 6}&\\
        & = \frac{16!}{6!\cdot 10!} & \\
        & = \frac{16\cdot 15\cdot 14\cdot 13\cdot 12\cdot 11\cdot 10!}{( 6\cdot 5\cdot 4\cdot 3\cdot 2\cdot)(10!)} & \\
        & = 4\cdot 14\cdot 13\cdot 11 &\\
        & = 8008 & \\
      \end{flalign*}
  \end{problem}

  \begin{problem}
    \section{Menchies}
      Since we are may choose multiple scoops of the same topping, and the order of choosing the
      toppings does not matter, we shall use multichoose to calculate the number of toppings
      required to create 90 masterpieces.

      \begin{flalign*}
        & \multichoose{n}{3} = 90 & \\
        & 90 = \binom{n - 1 + 3}{3} & \\
        & 90 = \frac{(n+2)!}{(n-1)!\cdot 3!} & \\
        & 90 = \frac{(n+2)(n+1)(n)}{3!} & \\
        & 540 = (n)(n+1)(n+2) & \\
        & n = 7.1842 &
      \end{flalign*}

      Rounding up, we shall need 8 toppings to be able to create 90 unique masterpieces.
  \end{problem}

  \begin{problem}
    \section{Robot Testing}
      If we select the recipe for the first robot as the first 10 bolts listed in order and the
      recipe for the second robot is the next 10 bolts listed in order, the given problem transforms
      into calculating the number of permutations of 20 bolts.

      \begin{flalign*}
        & = 20! & \\
        & = 2432902008176640000 \text{ must be tested.} & \\
      \end{flalign*}
  \end{problem}

  \begin{problem}
    \section{Elective Courses}
      Since the order of choice does not matter here, we use combinations. Here we assume no student
      can take more one elective, hence the number of students left in the pool after each round
      will reduce.

      \begin{flalign*}
        & = \binom{150}{30} \cdot \binom{120}{30} \cdot \binom{90}{45} \cdot \binom{45}{45} &\\
        & = \frac{150!}{30!\cdot 30!\cdot 45!\cdot 45!} & \\
        & = 56747866560602814669067058701650439905224429665398789752017071343634738487868082688000 &
      \end{flalign*}
  \end{problem}

\end{document}
